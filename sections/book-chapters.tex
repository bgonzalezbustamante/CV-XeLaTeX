%%%%%%%%%%%%%%%%%%%%%%%%%%%%%%%%%%%%%%%%%%%%%%%%%%

%% CV-XeLaTeX in English
%% Bastián González-Bustamante
%% University of Oxford
%% https://github.com/bgonzalezbustamante/CV-XeLaTeX

%% Based on the following repositories:
%% Awesome CV LaTeX Template for CV/Resume
%% https://github.com/posquit0/Awesome-CV
%% Bastián Gozález-Bustamante's CV-LaTeX Template
%% https://github.com/bgonzalezbustamante/CV-LaTeX
%% Carla Cisternas' CV-LuaLaTeX Template
%% https://github.com/carlacisternasg/CV-LuaLaTeX

%% LaTeX Project Public License v1.3c
%% https://github.com/bgonzalezbustamante/CV-XeLaTeX/blob/master/LICENSE.md

%%%%%%%%%%%%%%%%%%%%%%%%%%%%%%%%%%%%%%%%%%%%%%%%%%

\cvsection{Book Chapters}

\begin{publications}

\begin{benumerate}{12}

\item{\small González-Bustamante, B. (2018). Civil Service Models in Latin America. In A. Farazmand (ed.), {\itshape Global Encyclopedia of Public Administration, Public Policy, and Governance}. Cham: Springer. {\scshape doi:} \href{https://doi.org/10.1007/978-3-319-20928-9\_2699}{\textcolor{blue}{10.1007/978-3-319-20928-9\_2699}}. {\scshape \footnotesize SocArXiv:} \href{https://doi.org/10.31235/osf.io/mp4qd}{\textcolor{blue}{10.31235/osf.io/mp4qd}}.}\vspace{1mm}

\item{\small González-Bustamante, B., \& Olivares, A. (2018). Governmental Political Elite in Chile: Ministers Survival (1990-2014). In A. Codato \& F. Espinoza (eds.), {\itshape Las Élites en las Américas: Diferentes Perspectivas}. Curitiba: Editora da Universidade Federal do Paraná. {\footnotesize \scshape [in Spanish]}. {\scshape url:} \href{https://www.researchgate.net/publication/325699783_Elites_en_las_Americas_diferentes_perspectivas_Elites_in_the_Americas_Different_Perspectives}{\textcolor{blue}{www.researchgate.net}}.}\vspace{1mm}

\item{\small González-Bustamante, B., \& Barría, D. (2018). Expansion of the Public Sphere in Chile: Social Networks, Electoral Campaigns, and Digital Participation. In N. Del Valle (ed.), {\itshape Transformaciones de la Esfera Pública en Chile. Luchas Sociales, Espacio Público y Pluralismo Informativo}. Santiago: RIL Editores. {\footnotesize \scshape [in Spanish]}. {\scshape \footnotesize SocArXiv:} \href{https://doi.org/10.31235/osf.io/nkftb}{\textcolor{blue}{10.31235/osf.io/nkftb}}.}\vspace{1mm}

\item{\small González-Bustamante, B. (2018). Internet, Digital Social Networks Usage, and Participation in the Southern Cone. In P. Cottet (ed.), {\itshape Opinión Pública Contemporánea: Otras Posibilidades de Comprensión e Investigación}. Santiago: Social-Ediciones. {\footnotesize \scshape [in Spanish]}. {\scshape doi:} \href{https://doi.org/10.34720/2nd0-8t73}{\textcolor{blue}{10.34720/2nd0-8t73}}.}\vspace{1mm}

\item{\small Barría, D., González-Bustamante, B., \& Araya, J. E. (2017). Electronic Democracy and Digital Participation: Advances and Challenges. In J. R. Gil-García, J. I. Criado \&  J. C. Téllez (eds.), {\itshape Tecnologías de Información y Comunicación en la Administración Pública: Conceptos, Enfoques, Aplicaciones y Resultados}. Mexico City: INFOTEC. {\footnotesize \scshape [in Spanish]}. {\scshape url:} \href{https://www.researchgate.net/publication/321980289_Democracia_electronica_y_participacion_digital_Avances_y_desafios}{\textcolor{blue}{www.researchgate.net}}.}\vspace{1mm}

\item{\small González-Bustamante, B., \& Sazo, D. (2016). Viral Campaign. In I. Crespo {\itshape et al.} (eds.), {\itshape Diccionario Enciclopédico de Comunicación Política}. Madrid: Centro de Estudios Políticos y Constitucionales. {\footnotesize \scshape [in Spanish, 2nd edition]}.}\vspace{1mm}

\item{\small González-Bustamante, B., \& Sazo, D. (2016). Political Branding. In I. Crespo {\itshape et al.} (eds.), {\itshape Diccionario Enciclopédico de Comunicación Política}. Madrid: Centro de Estudios Políticos y Constitucionales. {\footnotesize \scshape [in Spanish, 2nd edition]}.}\vspace{1mm}

\item{\small González-Bustamante, B. (2016). Twitter Usage in Political Communication. In I. Crespo {\itshape et al.} (eds.), {\itshape Diccionario Enciclopédico de Comunicación Política}. Madrid: Centro de Estudios Políticos y Constitucionales. {\footnotesize \scshape [in Spanish, 2nd edition]}.}\vspace{1mm}

\item{\small González-Bustamante, B. (2014). New Media and Digital Politics: The Case of Chilean Primary Presidential Elections 2013. In G. Sibaja (ed.), {\itshape Nuevos Medios y Comunicación Política Digital}. San Jose: Fundación Educativa San Judas Tadeo. {\footnotesize \scshape [in Spanish]}.}\vspace{1mm}

\item{\small González-Bustamante, B. (2014). Digital Activism, Social Networks, and Brokerage. En S. Millaleo \& P. Cárcamo (eds.), {\itshape Mediaciones del Sistema Político frente al Activismo Digital}. Santiago: Fundación Democracia y Desarrollo. {\footnotesize \scshape [in Spanish]}. {\scshape url:} \href{https://www.researchgate.net/publication/321992867_Activismo_digital_redes_sociales_e_intermediacion}{\textcolor{blue}{www.researchgate.net}}.}\vspace{1mm}

\item{\small González-Bustamante, B. (2014). Social Networks as a Tool for Political Engage: Twitter and the Public Opinion Concept. In O. Avilés {\itshape et al.} (eds.), {\itshape Práctica Política y Medios Digitales}. Santiago: Instituto Igualdad. {\footnotesize \scshape [in Spanish]}.}\vspace{1mm}

\item{\small González-Bustamante, B., \& Henríquez, G. (2013). Chile: The Digital Campaign 2009-2010. In I. Crespo \& J. del Rey (eds.), {\itshape Comunicación Política \& Campa\~nas Electorales en América Latina}. Buenos Aires: Editorial Biblos. {\footnotesize \scshape [in Spanish].}}\vspace{1mm}

\end{benumerate}

\end{publications}
%% \pagebreak