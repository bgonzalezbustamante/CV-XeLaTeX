%%%%%%%%%%%%%%%%%%%%%%%%%%%%%%%%%%%%%%%%%%%%%%%%%%

%% CV-XeLaTeX in Spanish
%% Bastián González-Bustamante
%% University of Oxford
%% https://github.com/bgonzalezbustamante/CV-XeLaTeX

%% Based on the following repositories:
%% Awesome CV LaTeX Template for CV/Resume
%% https://github.com/posquit0/Awesome-CV
%% Bastián Gozález-Bustamante's CV-LaTeX Template
%% https://github.com/bgonzalezbustamante/CV-LaTeX
%% Carla Cisternas' CV-LuaLaTeX Template
%% https://github.com/carlacisternasg/CV-LuaLaTeX

%% Creative Commons Attribution 4.0 International License
%% https://github.com/bgonzalezbustamante/CV-XeLaTeX/blob/master/LICENSE.txt

%%%%%%%%%%%%%%%%%%%%%%%%%%%%%%%%%%%%%%%%%%%%%%%%%%

\cvsection{Publicaciones revisadas por pares}

\cvsubsection{ SciELO \& Latindex}

\begin{publications}

\begin{benumerate}{9}
\item{\small Gonz\'alez-Bustamante, B. (2016). Élites políticas, económicas e intelectuales: Una agenda de investigación creciente para la ciencia política. {\itshape Pol\'itica, Revista de Ciencia Pol\'itica, 54}(1), 7--17. \\ {\scshape url}: \href{https://revistapolitica.uchile.cl/index.php/RP/article/view/42690}{\textcolor{blue}{https://revistapolitica.uchile.cl}}}\vspace{1mm}

\item{\small Gonz\'alez-Bustamante, B., \& Cisternas, C. (2016). Élites políticas en el poder legislativo chileno: La Cámara de Diputados (1990-2014). {\itshape Pol\'itica, Revista de Ciencia Pol\'itica, 54}(1), 19--52. {\scshape url}: \href{https://revistapolitica.uchile.cl/index.php/RP/article/view/42691}{\textcolor{blue}{https://revistapolitica.uchile.cl}}}\vspace{1mm}

\item{\small Gonz\'alez-Bustamante, B. (2015). Evaluando Twitter como indicador de opinión pública: Una mirada al arribo de Bachelet a la presidencial chilena 2013. {\itshape Revista SAAP, 9}(1), 119--141. {\scshape url}: \href{http://ref.scielo.org/dwzhns}{\textcolor{blue}{http://ref.scielo.org}}} \vspace{1mm}

\item{\small Gonz\'alez-Bustamante, B. (2015). Éxito electoral y gasto en campaña en las elecciones de senadores y diputados en Chile 2013. {\itshape Pol\'iticas P\'ublicas, 8}(1), 21--35. {\scshape url}: \href{http://www.revistas.usach.cl/ojs/index.php/politicas/article/view/2182}{\textcolor{blue}{www.revistas.usach.cl}}} \vspace{1mm}

\item{\small Gonz\'alez-Bustamante, B. (2014). Elección directa de consejeros regionales 2013. Rendimiento del capital político, familiar y económico en una nueva arena electoral en Chile. {\itshape Pol\'itica, Revista de Ciencia Pol\'itica, 52}(2), 49--91. {\scshape url}: \href{https://revistapolitica.uchile.cl/index.php/RP/article/view/36137}{\textcolor{blue}{https://revistapolitica.uchile.cl}}} \vspace{1mm}

\item{\small Olivares, A. {\itshape et al.} (2014). Los think tanks en el gabinete: Una exploración del caso chileno (2006-2014). {\itshape Revista de Sociolog\'ia,} (29), 37--54. {\scshape url}: \href{https://revistadesociologia.uchile.cl/index.php/RDS/article/view/36177}{\textcolor{blue}{https://revistadesociologia.uchile.cl}}} \vspace{1mm}

\item{\small Gonz\'alez-Bustamante, B. (2013). Factores de acceso y permanencia de la élite política gubernamental en Chile (1990-2010). {\itshape Pol\'itica, Revista de Ciencia Pol\'itica, 51}(1), 119--153. {\scshape url}: \href{https://revistapolitica.uchile.cl/index.php/RP/article/view/27436}{\textcolor{blue}{https://revistapolitica.uchile.cl}}} \vspace{1mm}

\item{\small Gonz\'alez-Bustamante, B. (2013). El estudio de las élites en Chile: Aproximaciones conceptuales y metodológicas. {\itshape Intersticios Sociales,} (6), 1--20. {\scshape url}: \href{http://ref.scielo.org/zrnp2k}{\textcolor{blue}{http://ref.scielo.org}}} \vspace{1mm}

\item{\small Gonz\'alez-Bustamante, B., \& Henr\'iquez, G. (2012). Campañas digitales: ¿Branding o participación política? {\itshape M\'as Poder Local,} (12), 32--39. {\scshape url}: \href{https://www.researchgate.net/publication/260517478_Campanas_digitales_Branding_o_participacion_politica_El_rol_de_las_redes_sociales_en_la_ultima_campana_presidencial_chilena}{\textcolor{blue}{www.researchgate.net}}} \vspace{1mm}
\end{benumerate}

\end{publications}
