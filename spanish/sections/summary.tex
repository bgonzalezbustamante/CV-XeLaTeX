%%%%%%%%%%%%%%%%%%%%%%%%%%%%%%%%%%%%%%%%%%%%%%%%%%

%% CV-XeLaTeX in Spanish
%% Bastián González-Bustamante
%% University of Oxford
%% https://github.com/bgonzalezbustamante/CV-XeLaTeX

%% Based on the following repositories:
%% Awesome CV LaTeX Template for CV/Resume
%% https://github.com/posquit0/Awesome-CV
%% Bastián Gozález-Bustamante's CV-LaTeX Template
%% https://github.com/bgonzalezbustamante/CV-LaTeX
%% Carla Cisternas' CV-LuaLaTeX Template
%% https://github.com/carlacisternasg/CV-LuaLaTeX

%% LaTeX Project Public License v1.3c
%% https://github.com/bgonzalezbustamante/CV-XeLaTeX/blob/master/LICENSE.md

%%%%%%%%%%%%%%%%%%%%%%%%%%%%%%%%%%%%%%%%%%%%%%%%%%

\vspace{3mm}
\cvsection{Reseña}

\begin{cvparagraph}

Estoy completando mi tesis doctoral en el Departamento de Ciencia Política y Relaciones Internacionales y en St Hilda’s College en la Universidad de Oxford, Reino Unido.  Antes de comenzar mis estudios doctorales en octubre de 2019, obtuve un Magíster en Ciencia Política (distinción máxima) y una Licenciatura en Ciencias Políticas y Gubernamentales con mención en Gestión Pública (distinción), ambos en la Universidad de Chile. Además, me desempeñé como profesor adjunto y docente en la Universidad de Santiago y Universidad de Chile y como consultor del Programa de las Naciones Unidas para el Desarrollo (PNUD).

Mi investigación doctoral es supervisada por la Profesora Petra Schleiter y se centra en las causas y consecuencias de la rotación ministerial en 12 países presidenciales latinoamericanos desde mediados de los años 70 hasta la fecha. Mi proyecto integra elementos de la teoría principal-agente para evaluar los incentivos y las estrategias específicas de los actores en los sistemas presidenciales, considerando las características institucionales del régimen. En mi proceso de recopilación de datos, apliqué técnicas de reconocimiento óptico de caracteres (OCR), procesamiento del lenguaje natural (NLP) y aprendizaje automático a casi 50 años de archivos de informes de prensa. Además, mi estrategia empírica emplea métodos de emparejamiento de puntuación de propensión, modelos semiparamétricos de riesgos competitivos y regresiones de variables instrumentales.

Mi investigación es financiada por la Agencia Nacional de Investigación y Desarrollo (ANID) de Chile, el Fondo Muriel Wise de St Hilda's College y el Premio de Postgrado y Postdoctorado de la Sociedad de Estudios Latinoamericanos (SLAS).
\vspace{1mm}
\end{cvparagraph}