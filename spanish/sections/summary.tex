%%%%%%%%%%%%%%%%%%%%%%%%%%%%%%%%%%%%%%%%%%%%%%%%%%

%% CV-XeLaTeX in Spanish
%% Bastián González-Bustamante
%% University of Oxford
%% https://github.com/bgonzalezbustamante/CV-XeLaTeX

%% Based on the following repositories:
%% Awesome CV LaTeX Template for CV/Resume
%% https://github.com/posquit0/Awesome-CV
%% Bastián Gozález-Bustamante's CV-LaTeX Template
%% https://github.com/bgonzalezbustamante/CV-LaTeX
%% Carla Cisternas' CV-LuaLaTeX Template
%% https://github.com/carlacisternasg/CV-LuaLaTeX

%% Creative Commons Attribution 4.0 International License
%% https://github.com/bgonzalezbustamante/CV-XeLaTeX/blob/master/LICENSE.txt

%%%%%%%%%%%%%%%%%%%%%%%%%%%%%%%%%%%%%%%%%%%%%%%%%%

\vspace{3mm}
\cvsection{Reseña}

\begin{cvparagraph}

Estoy completando mi tesis doctoral en el Departamento de Ciencia Política y Relaciones Internacionales y en St Hilda’s College en la Universidad de Oxford, Reino Unido. Actualmente me encuentro establecido en Oxford, trabajando en mi investigación. Además, soy Instructor en el Departamento de Gestión y Políticas Públicas de la Facultad de Administración y Economía de la Universidad de Santiago de Chile.

Antes de comenzar mis estudios doctorales, en octubre de 2019, obtuve un Magíster en Ciencia Política (distinción máxima) y una Licenciatura en Ciencias Políticas y Gubernamentales con mención en Gestión Pública (distinción), ambos en la Universidad de Chile. Además, me desempeñé como profesor en la Universidad de Chile y como consultor del Programa de las Naciones Unidas para el Desarrollo (PNUD).

Mi investigación doctoral es supervisada por la Profesora Petra Schleiter y se enfoca en la estabilidad de los gabinetes en las democracias presidenciales latinoamericanas. En mi proyecto doctoral combino elementos de la teoría principal-agente para evaluar los incentivos y estrategias de los actores en los sistemas presidenciales. Mi estrategia empírica emplea métodos de puntuación de propensión y emparejamiento, modelos de riesgos competitivos y modelos de variables instrumentales. Mi investigación es financiada por la Agencia Nacional de Investigación y Desarrollo de Chile.
\vspace{1mm}
\end{cvparagraph}
