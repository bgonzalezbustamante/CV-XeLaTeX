%%%%%%%%%%%%%%%%%%%%%%%%%%%%%%%%%%%%%%%%%%%%%%%%%%

%% CV-XeLaTeX in Spanish
%% Bastián González-Bustamante
%% University of Oxford
%% https://github.com/bgonzalezbustamante/CV-XeLaTeX

%% Based on the following repositories:
%% Awesome CV LaTeX Template for CV/Resume
%% https://github.com/posquit0/Awesome-CV
%% Bastián Gozález-Bustamante's CV-LaTeX Template
%% https://github.com/bgonzalezbustamante/CV-LaTeX
%% Carla Cisternas' CV-LuaLaTeX Template
%% https://github.com/carlacisternasg/CV-LuaLaTeX

%% LaTeX Project Public License v1.3c
%% https://github.com/bgonzalezbustamante/CV-XeLaTeX/blob/master/LICENSE.md

%%%%%%%%%%%%%%%%%%%%%%%%%%%%%%%%%%%%%%%%%%%%%%%%%%

\cvsection{Artículos revisados por pares}

\cvsubsection{Social Sciences Citation Index (WoS-SSCI, ex ISI)}

\begin{publications}

\begin{benumerate}{9}

\item{González-Bustamante, B. (2022). Ministerial stability during presidential approval crises: The moderating effect of ministers' attributes on dismissals in Brazil and Chile. {\itshape The British Journal of Politics and International Relations}. OnlineFirst. {\scshape doi:} \href{https://doi.org/10.1177/13691481221124850}{\textcolor{blue}{10.1177/13691481221124850}}.}\vspace{1mm}

\item{Fleming, T. G., González-Bustamante, B., \& Schleiter, P. (2022). Cabinet Reshuffles and Parliamentary No-Confidence Motions. {\itshape Government and Opposition}. FirstView. {\scshape doi:} \href{https://doi.org/10.1017/gov.2022.23}{\textcolor{blue}{10.1017/gov.2022.23}}.}\vspace{1mm}

\item{Orchard, X., \& González-Bustamante, B. (2022). Power Hierarchies and Visibility in the News: Exploring Determinants of Politicians’ Presence and Prominence in the Chilean Press (1991-2019). {\itshape The International Journal of Press/Politics}. OnlineFirst. {\scshape doi:} \href{https://doi.org/10.1177/19401612221089482}{\textcolor{blue}{10.1177/19401612221089482}}.}

\item{Cuevas, C. et al. (2021). Motivación de servicio público entre funcionarios públicos chilenos. {\itshape Revista del CLAD Reforma y Democracia}, 81, 105--138. {\scshape url:} \href{https://clad.org/documentacion/revista-clad/articulos-publicados/081-noviembre-2021/}{\textcolor{blue}{https://clad.org}}. {\scshape \footnotesize SocArXiv:} \href{https://doi.org/10.31235/osf.io/p963q}{\textcolor{blue}{10.31235/osf.io/p963q}}.}\vspace{1mm}

\item{González-Bustamante, B. (2021). Evolution and early government responses to COVID-19 in South America. {\itshape World Development, 137}, 105180. {\scshape doi:} \href{https://doi.org/10.1016/j.worlddev.2020.105180}{\textcolor{blue}{10.1016/j.worlddev.2020.105180}}.}\vspace{1mm}

\item{González-Bustamante, B. (2020). The Politics-Administration Dichotomy: A Case Study of the Chilean Executive During the Democratic Post-Transition. {\itshape Bulletin of Latin American Research, 39}(5), 582--597. {\scshape doi}: \\ \href{https://doi.org/10.1111/blar.13044}{\textcolor{blue}{10.1111/blar.13044}}. {\scshape \footnotesize SocArXiv:} \href{https://doi.org/10.31235/osf.io/d52au}{\textcolor{blue}{10.31235/osf.io/d52au}}.}\vspace{1mm}

\item{González-Bustamante, B., Carvajal, A., \& González, A. (2020). Determinantes del gobierno electrónico en las municipalidades: Evidencia del caso chileno. {\itshape Gestión y Política Pública, XXIX}(1), 97--129. {\scshape doi:} \\ \href{http://dx.doi.org/10.29265/gypp.v29i1.658}{\textcolor{blue}{10.29265/gypp.v29i1.658}}. {\scshape \footnotesize SocArXiv:} \href{https://doi.org/10.31235/osf.io/fze3x}{\textcolor{blue}{10.31235/osf.io/fze3x}}.} \vspace{1mm}

\item{González-Bustamante, B. (2019). Brechas, representación y congruencia élite-ciudadanía en Chile y Uruguay. {\itshape Convergencia. Revista de Ciencias Sociales}, 80, 1--27. {\scshape doi}: \href{https://doi.org/10.29101/crcs.v26i80.11097}{\textcolor{blue}{10.29101/crcs.v26i80.11097}}. {\scshape \footnotesize SocArXiv:} \\ \href{https://doi.org/10.31235/osf.io/cqym8}{\textcolor{blue}{10.31235/osf.io/cqym8}}.}\vspace{1mm} %% {\scshape url:} \href{https://www.redalyc.org/jatsRepo/105/10559568002/index.html}{\textcolor{blue}{www.redalyc.org}}

\item{González-Bustamante, B., \& Garrido-Vergara, L. (2018). Socialización, trayectorias y poscarrera de ministros en Chile, 1990-2010. {\itshape Política y Gobierno, XXV}(1), 31--64. {\scshape url:} \href{http://www.politicaygobierno.cide.edu/index.php/pyg/article/view/1080}{\textcolor{blue}{www.politicaygobierno.cide.edu}}.}\vspace{1mm}

\end{benumerate}

\end{publications}

\cvsubsection{Scopus \& Emerging Sources Citation Index (WoS-ESCI)}

\begin{publications}

\begin{benumerate}{10}

\item{González-Bustamante, B., \& Aguilar, D. (2023). Territorial patterns of open e-government: evidence from Chilean municipalities. {\itshape Political Research Exchange, 5}(1). {\scshape doi:} \href{https://doi.org/10.1080/2474736X.2023.2194369}{\textcolor{blue}{10.1080/2474736X.2023.2194369}}. {\scshape \footnotesize SocArXiv:} \href{https://doi.org/10.31235/osf.io/gt8a5}{\textcolor{blue}{10.31235/osf.io/gt8a5}}.}\vspace{1mm}

\item{Cisternas, C., \& González-Bustamante, B. (2022). Ciencias sociales en contextos de represión: Análisis bibliométrico de la producción histórica de la Corporación de Estudios para Latinoamérica, Chile (1979-1989). {\itshape e-Ciencias de la Información, 12}(2), 1--18. {\scshape doi:} \href{https://doi.org/10.15517/eci.v12i2.50078}{\textcolor{blue}{10.15517/eci.v12i2.50078}}. {\scshape \footnotesize SocArXiv:} \href{https://doi.org/10.31235/osf.io/2jrvm}{\textcolor{blue}{10.31235/osf.io/2jrvm}}.}\vspace{1mm} %% {\scshape url:} \href{https://revistas.ucr.ac.cr/index.php/eciencias/article/view/50078}{\textcolor{blue}{https://revistas.ucr.ac.cr}}

\item{González-Bustamante, B., Astete, M., \& Orvenes, B. (2020). Altos directivos públicos: Un nuevo conjunto de datos de miembros del servicio civil chileno. {\itshape Revista de Gestión Pública, IX}(2), 151--169. {\scshape doi}: \\ \href{https://doi.org/10.22370/rgp.2020.9.2.2920}{\textcolor{blue}{10.22370/rgp.2020.9.2.2920}}. {\scshape \footnotesize SocArXiv:} \href{https://doi.org/10.31235/osf.io/vshcz}{\textcolor{blue}{10.31235/osf.io/vshcz}}.}\vspace{1mm}

\item{Olivares, A. et al. (2020). Nuevos desafíos, enfoques y perspectivas para estudiar élites políticas. {\itshape Iberoamericana, XX}(74), 229--259. {\scshape doi}: \href{https://doi.org/10.18441/ibam.20.2020.74.229-259}{\textcolor{blue}{10.18441/ibam.20.2020.74.229-259}}. {\scshape \footnotesize SocArXiv:} \href{https://doi.org/10.31235/osf.io/syqu4}{\textcolor{blue}{10.31235/osf.io/syqu4}}.}\vspace{1mm}

\item{Maillet, A., González-Bustamante, B., \& Olivares, A. (2019). Public-Private Circulation and the Revolving Door in the Chilean Executive Branch (2000-2014). {\itshape Latin American Business Review, 20}(4), 367--387. {\scshape doi}: \\ \href{https://doi.org/10.1080/10978526.2019.1652099}{\textcolor{blue}{10.1080/10978526.2019.1652099}}.}\vspace{1mm}

\item{Barría, D., González-Bustamante, B., \& Cisternas, C. (2019). La literatura sobre gobierno abierto en español. Análisis sobre las dinámicas de producción y citación. {\itshape Nóesis. Revista de Ciencias Sociales y Humanidades, 28}(56), 22--42. {\scshape doi}: \href{http://dx.doi.org/10.20983/noesis.2019.2.3}{\textcolor{blue}{10.20983/noesis.2019.2.3}}.}\vspace{1mm}

\item{Del Valle, N., \& González-Bustamante, B. (2018). Agenda política, periodismo y medios digitales en Chile: Notas de investigación sobre pluralismo informativo. {\itshape Perspectivas de la Comunicación, 11}(1), 291--326. {\scshape url:} \href{https://revistas.ufro.cl/ojs/index.php/perspectivas/article/view/1146}{\textcolor{blue}{https://revistas.ufro.cl}}.}\vspace{1mm}

\item{González-Bustamante, B., \& Olivares, A. (2016). Cambios de gabinete y supervivencia de los ministros en Chile durante los gobiernos de la Concertación (1990-2010). {\itshape Colombia Internacional}, 87, 81--108. {\scshape doi}: \\ \href{https://doi.org/10.7440/colombiaint87.2016.04}{\textcolor{blue}{10.7440/colombiaint87.2016.04}}.}\vspace{1mm}

\item{González-Bustamante, B. et al. (2016). Servicio civil en Chile, análisis de los directivos de primer nivel jerárquico (2003-13). {\itshape Revista de Administra\c{c}\~ao Pública, 50}(1), 59--79. {\scshape doi}: \href{http://dx.doi.org/10.1590/0034-7612145767}{\textcolor{blue}{10.1590/0034-7612145767}}.} \vspace{1mm}

\item{González-Bustamante, B., \& Olivares, A. (2015). Rotación de subsecretarios en Chile: Una exploración de la segunda línea gubernamental, 1990-2014. {\itshape Revista de Gestión Pública, IV}(2), 151--190. {\scshape doi}: \\ \href{https://doi.org/10.22370/rgp.2015.4.2.2230}{\textcolor{blue}{10.22370/rgp.2015.4.2.2230}}.} \vspace{1mm}

\end{benumerate}

\end{publications}

\cvsubsection{ SciELO \& Latindex}

\begin{publications}

\begin{benumerate}{14}

\item{González-Bustamante. B. (2022). Métodos cuantitativos para estudiar a las élites: Aplicaciones prácticas, sesgos y potencialidades. {\itshape Revista Chilena de Derecho y Ciencia Política, 13}(2), 12--44. {\scshape doi:} \href{https://doi.org/10.7770/rchdcp-V13N2-art2907}{\textcolor{blue}{10.7770/rchdcp-V13N2-art2907}}. {\scshape \footnotesize SocArXiv:} \href{https://doi.org/10.31235/osf.io/5m2ur}{\textcolor{blue}{10.31235/osf.io/5m2ur}}.}\vspace{1mm}

\item{González-Bustamante, B. (2021). Sofisticación, participación y compromiso político en América Latina. {\itshape Tufte Working Papers}, 2, 1--21. {\scshape doi:} \href{https://doi.org/10.5281/zenodo.6739833}{\textcolor{blue}{10.5281/zenodo.6739833}}. {\scshape \footnotesize SocArXiv:} \href{https://doi.org/10.31235/osf.io/r4pj8}{\textcolor{blue}{10.31235/osf.io/r4pj8}}.}\vspace{1mm}

\item{González-Bustamante, B., \& Luci, F. (2021). Élites políticas en América Latina: Socialización, trayectorias y capitales. {\itshape Pléyade, Revista de Humanidades y Ciencias Sociales}, 28. 21--32. {\scshape url:} \href{http://www.revistapleyade.cl/index.php/OJS/article/view/359}{\textcolor{blue}{www.revistapleyade.cl}}. \\ {\scshape \footnotesize SocArXiv:} \href{https://doi.org/10.31235/osf.io/wud59}{\textcolor{blue}{10.31235/osf.io/wud59}}.}\vspace{1mm}

\item{González-Bustamante, B. (2021).  Hibridación digital en el Cono Sur: Consumo de medios tradicionales, digitales y uso de redes sociales. {\itshape Comunifé: Revista de Comunicación Social}, 21, 31--39. \\ {\scshape url:} \href{https://revistas.unife.edu.pe/index.php/comunife/article/view/2580/}{\textcolor{blue}{https://revistas.unife.edu.pe}}.}\vspace{1mm}

\item{González-Bustamante, B., \& Cisternas, C. (2020). Aplicación de ForceAtlas2, un algoritmo de diseño gráfico continuo, para el estudio de las élites. {\itshape Tufte Working Papers}, 1, 1--15. {\scshape doi:} \href{https://doi.org/10.5281/zenodo.6739266}{\textcolor{blue}{10.5281/zenodo.6739266}}. {\scshape \footnotesize SocArXiv:} \href{https://doi.org/10.31235/osf.io/gxrkc}{\textcolor{blue}{10.31235/osf.io/gxrkc}}.}\vspace{1mm}

\item{González-Bustamante, B., \& Cisternas, C. (2016). Élites políticas en el poder legislativo chileno: la Cámara de Diputados (1990-2014). {\itshape Política, Revista de Ciencia Política, 54}(1), 19--52. {\scshape url:} \href{https://revistapolitica.uchile.cl/index.php/RP/article/view/42691}{\textcolor{blue}{https://revistapolitica.uchile.cl}}.}\vspace{1mm}

\item{González-Bustamante, B. (2016). Élites políticas, económicas e intelectuales: una agenda de investigación creciente para la ciencia política. {\itshape Política, Revista de Ciencia Política, 54}(1), 7--17. {\scshape url:} \\ \href{https://revistapolitica.uchile.cl/index.php/RP/article/view/42690}{\textcolor{blue}{https://revistapolitica.uchile.cl}}.}\vspace{1mm}

\item{González-Bustamante, B. (2015). Éxito electoral y gasto en campaña en las elecciones de senadores y diputados en Chile 2013. {\itshape Políticas Públicas, 8}(1), 21--35. {\scshape url:} \href{http://www.revistas.usach.cl/ojs/index.php/politicas/article/view/2182}{\textcolor{blue}{www.revistas.usach.cl}}.} \vspace{1mm}

\item{González-Bustamante, B. (2015). Evaluando Twitter como indicador de opinión pública: una mirada al arribo de Bachelet a la presidencial chilena 2013. {\itshape Revista SAAP, 9}(1), 119--141. {\scshape url:} \href{https://www.redalyc.org/articulo.oa?id=387142733006}{\textcolor{blue}{www.redalyc.org}}.} \vspace{1mm}

\item{Olivares, A. et al. (2014). Los think tanks en el gabinete: una exploración del caso chileno (2006-2014). {\itshape Revista de Sociología,} 29, 37--54. {\scshape doi}: \href{https://doi.org/10.5354/0719-529X.2014.36177}{\textcolor{blue}{10.5354/0719-529X.2014.36177}}.} \vspace{1mm} %% {\scshape url:} \href{https://revistadesociologia.uchile.cl/index.php/RDS/article/view/36177}{\textcolor{blue}{https://revistadesociologia.uchile.cl}}

\item{González-Bustamante, B. (2014). Elección directa de consejeros regionales 2013. Rendimiento del capital político, familiar y económico en una nueva arena electoral en Chile. {\itshape Política, Revista de Ciencia Política, 52}(2), 49--91. {\scshape url:} \href{https://revistapolitica.uchile.cl/index.php/RP/article/view/36137}{\textcolor{blue}{https://revistapolitica.uchile.cl}}} \vspace{1mm}

\item{González-Bustamante, B. (2013). El estudio de las élites en Chile: aproximaciones conceptuales y metodológicas. {\itshape Intersticios Sociales,} 6, 1--20. {\scshape url:} \href{https://www.redalyc.org/articulo.oa?id=421739499004}{\textcolor{blue}{www.redalyc.org}}.} \vspace{1mm}

\item{González-Bustamante, B. (2013). Factores de acceso y permanencia de la élite política gubernamental en Chile (1990-2010). {\itshape Política, Revista de Ciencia Política, 51}(1), 119--153. {\scshape doi}: \href{https://doi.org/10.5354/0716-1077.2013.27436}{\textcolor{blue}{10.5354/0716-1077.2013.27436}}.} \vspace{1mm} %% {\scshape url:} \href{https://revistapolitica.uchile.cl/index.php/RP/article/view/27436}{\textcolor{blue}{https://revistapolitica.uchile.cl}}

\item{González-Bustamante, B., \& Henríquez, G. (2012). Campañas digitales: ¿Branding o participación política? {\itshape Más Poder Local,} 12, 32--39. {\scshape url:} \href{https://dialnet.unirioja.es/servlet/articulo?codigo=4013864}{\textcolor{blue}{https://dialnet.unirioja.es}}.} \vspace{1mm} %% {\scshape url:} \href{https://www.researchgate.net/publication/260517478_Campanas_digitales_Branding_o_participacion_politica_El_rol_de_las_redes_sociales_en_la_ultima_campana_presidencial_chilena}{\textcolor{blue}{www.researchgate.net}}

\end{benumerate}

\end{publications}
%% \pagebreak
