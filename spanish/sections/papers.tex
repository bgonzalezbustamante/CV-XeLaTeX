%%%%%%%%%%%%%%%%%%%%%%%%%%%%%%%%%%%%%%%%%%%%%%%%%%

%% CV-XeLaTeX in Spanish
%% Bastián González-Bustamante
%% University of Oxford
%% https://github.com/bgonzalezbustamante/CV-XeLaTeX

%% Based on the following repositories:
%% Awesome CV LaTeX Template for CV/Resume
%% https://github.com/posquit0/Awesome-CV
%% Bastián Gozález-Bustamante's CV-LaTeX Template
%% https://github.com/bgonzalezbustamante/CV-LaTeX
%% Carla Cisternas' CV-LuaLaTeX Template
%% https://github.com/carlacisternasg/CV-LuaLaTeX

%% Creative Commons Attribution 4.0 International License
%% https://github.com/bgonzalezbustamante/CV-XeLaTeX/blob/master/LICENSE.txt

%%%%%%%%%%%%%%%%%%%%%%%%%%%%%%%%%%%%%%%%%%%%%%%%%%

\cvsection{Publicaciones revisadas por pares}

\cvsubsection{Social Sciences Citation Index (WoS-ISI/SCCI)}

\begin{publications}

\begin{benumerate}{5}
\item{\small Gonz\'alez-Bustamante, B. ({\itshape próximamente}). Evolution and Early Government Responses to COVID-19 in South America. {\itshape World Development}.}\vspace{1mm}

\item{\small Gonz\'alez-Bustamante, B., Carvajal, A., \& Gonz\'alez, A. (2020). Determinantes del gobierno electrónico en las municipalidades: Evidencia del caso chileno. {\itshape Gesti\'on y Pol\'itica P\'ublica, XXIX}(1), 97--129. \\ {\scshape doi:} \href{http://dx.doi.org/10.29265/gypp.v29i1.658}{\textcolor{blue}{10.29265/gypp.v29i1.658}}. {\scshape {\footnotesize SocArXiv} doi}: \href{https://doi.org/10.31235/osf.io/fze3x}{\textcolor{blue}{10.31235/osf.io/fze3x}}} \vspace{1mm}

\item{\small Gonz\'alez-Bustamante, B. (2019). The Politics-Administration Dichotomy: A Case Study of the Chilean Executive During the Democratic Post-Transition. {\itshape Bulletin of Latin American Research}. Early View. \\ {\scshape doi}: \href{https://doi.org/10.1111/blar.13044}{\textcolor{blue}{10.1111/blar.13044}}}\vspace{1mm}

\item{\small Gonz\'alez-Bustamante, B. (2019). Brechas, representación y congruencia élite-ciudadanía en Chile y Uruguay. {\itshape Convergencia. Revista de Ciencias Sociales}, (80), 1--27. {\scshape doi}: \href{https://doi.org/10.29101/crcs.v26i80.11097}{\textcolor{blue}{10.29101/crcs.v26i80.11097}}. \\ {\scshape {\footnotesize SocArXiv} doi}: \href{https://doi.org/10.31235/osf.io/cqym8}{\textcolor{blue}{10.31235/osf.io/cqym8}}}\vspace{1mm}

\item{\small Gonz\'alez-Bustamante, B., \& Garrido-Vergara, L. (2018). Socialización, trayectorias y poscarrera de ministros en Chile, 1990-2010. {\itshape Pol\'itica y Gobierno, XXV}(1), 31--64. {\scshape url}: \href{http://www.politicaygobierno.cide.edu/index.php/pyg/article/view/1080}{\textcolor{blue}{www.politicaygobierno.cide.edu}}}\vspace{1mm}
\end{benumerate}

\end{publications}

\cvsubsection{Scopus y Emerging Sources Citation Index (WoS-ISI/ESCI)}

\begin{publications}

\begin{benumerate}{7}
\item{\small Olivares, A. {\itshape et al.} (2020). Nuevos desafíos, enfoques y perspectivas para estudiar élites políticas. {\itshape Iberoamericana, XX}(74), 229--259. {\scshape url}: \href{https://journals.iai.spk-berlin.de/index.php/iberoamericana/article/view/2736}{\textcolor{blue}{https://journals.iai.spk-berlin.de}}. {\scshape {\footnotesize SocArXiv} doi}: \href{https://doi.org/10.31235/osf.io/syqu4}{\textcolor{blue}{10.31235/osf.io/syqu4}}}\vspace{1mm}

\item{\small Maillet, A., Gonz\'alez-Bustamante, B., \& Olivares, A. (2019). Public-Private Circulation and the Revolving Door in the Chilean Executive Branch (2000-2014). {\itshape Latin American Business Review, 20}(4), 367--387. \\ {\scshape doi}: \href{https://doi.org/10.1080/10978526.2019.1652099}{\textcolor{blue}{10.1080/10978526.2019.1652099}}}\vspace{1mm}

\item{\small Barr\'ia, D., Gonz\'alez-Bustamante, B., \& Cisternas, C. (2019). La literatura sobre gobierno abierto en español. Análisis sobre las dinámicas de producción y citación. {\itshape N\'oesis. Revista de Ciencias Sociales y Humanidades, 28}(56), 22--42.. {\scshape doi}: \href{http://dx.doi.org/10.20983/noesis.2019.2.3}{\textcolor{blue}{10.20983/noesis.2019.2.3}}}\vspace{1mm}

\item{\small Del Valle, N., \& Gonz\'alez-Bustamante, B. (2018). Agenda política, periodismo y medios digitales en Chile. Notas de investigación sobre pluralismo informativo. {\itshape Perspectivas de la Comunicaci\'on, 11}(1), 291--326. {\scshape url}: \href{http://revistas.ufro.cl/ojs/index.php/perspectivas/article/view/1146}{\textcolor{blue}{http://revistas.ufro.cl}}}\vspace{1mm}

\item{\small Gonz\'alez-Bustamante, B., \& Olivares, A. (2016). Cambios de gabinete y supervivencia de los ministros en Chile durante los gobiernos de la Concertación (1990-2010). {\itshape Colombia Internacional}, (87), 81--108. \\ {\scshape doi}: \href{https://doi.org/10.7440/colombiaint87.2016.04}{\textcolor{blue}{10.7440/colombiaint87.2016.04}}}\vspace{1mm}

\item{\small Gonz\'alez-Bustamante, B. {\itshape et al.} (2016). Servicio civil en Chile, análisis de los directivos de primer nivel jerárquico (2003-13). {\itshape Revista de Administra\c{c}\~ao P\'ublica, 50}(1), 59--79. {\scshape doi}: \href{http://dx.doi.org/10.1590/0034-7612145767}{\textcolor{blue}{10.1590/0034-7612145767}}} \vspace{1mm}

\item{\small Gonz\'alez-Bustamante, B., \& Olivares, A. (2015). Rotación de subsecretarios en Chile: Una exploración de la segunda línea gubernamental, 1990-2014. {\itshape Revista de Gesti\'on P\'ublica, IV}(2), 151--190. \\ {\scshape url}: \href{https://www.researchgate.net/publication/321977889_Rotacion_de_subsecretarios_en_Chile_una_exploracion_de_la_segunda_linea_gubernamental_1990-2014}{\textcolor{blue}{www.researchgate.net}}} \vspace{1mm}
\end{benumerate}

\end{publications}

\cvsubsection{ SciELO y Latindex}

\begin{publications}

\begin{benumerate}{9}
\item{\small Gonz\'alez-Bustamante, B. (2016). Élites políticas, económicas e intelectuales: Una agenda de investigación creciente para la ciencia política. {\itshape Pol\'itica, Revista de Ciencia Pol\'itica, 54}(1), 7--17. \\ {\scshape url}: \href{https://revistapolitica.uchile.cl/index.php/RP/article/view/42690}{\textcolor{blue}{https://revistapolitica.uchile.cl}}}\vspace{1mm}

\item{\small Gonz\'alez-Bustamante, B., \& Cisternas, C. (2016). Élites políticas en el poder legislativo chileno: La Cámara de Diputados (1990-2014). {\itshape Pol\'itica, Revista de Ciencia Pol\'itica, 54}(1), 19--52. {\scshape url}: \href{https://revistapolitica.uchile.cl/index.php/RP/article/view/42691}{\textcolor{blue}{https://revistapolitica.uchile.cl}}}\vspace{1mm}

\item{\small Gonz\'alez-Bustamante, B. (2015). Evaluando Twitter como indicador de opinión pública: Una mirada al arribo de Bachelet a la presidencial chilena 2013. {\itshape Revista SAAP, 9}(1), 119--141. {\scshape url}: \href{http://ref.scielo.org/dwzhns}{\textcolor{blue}{http://ref.scielo.org}}} \vspace{1mm}

\item{\small Gonz\'alez-Bustamante, B. (2015). Éxito electoral y gasto en campaña en las elecciones de senadores y diputados en Chile 2013. {\itshape Pol\'iticas P\'ublicas, 8}(1), 21--35. {\scshape url}: \href{http://www.revistas.usach.cl/ojs/index.php/politicas/article/view/2182}{\textcolor{blue}{www.revistas.usach.cl}}} \vspace{1mm}

\item{\small Gonz\'alez-Bustamante, B. (2014). Elección directa de consejeros regionales 2013. Rendimiento del capital político, familiar y económico en una nueva arena electoral en Chile. {\itshape Pol\'itica, Revista de Ciencia Pol\'itica, 52}(2), 49--91. {\scshape url}: \href{https://revistapolitica.uchile.cl/index.php/RP/article/view/36137}{\textcolor{blue}{https://revistapolitica.uchile.cl}}} \vspace{1mm}

\item{\small Olivares, A. {\itshape et al.} (2014). Los think tanks en el gabinete: Una exploración del caso chileno (2006-2014). {\itshape Revista de Sociolog\'ia,} (29), 37--54. {\scshape url}: \href{https://revistadesociologia.uchile.cl/index.php/RDS/article/view/36177}{\textcolor{blue}{https://revistadesociologia.uchile.cl}}} \vspace{1mm}

\item{\small Gonz\'alez-Bustamante, B. (2013). Factores de acceso y permanencia de la élite política gubernamental en Chile (1990-2010). {\itshape Pol\'itica, Revista de Ciencia Pol\'itica, 51}(1), 119--153. {\scshape url}: \href{https://revistapolitica.uchile.cl/index.php/RP/article/view/27436}{\textcolor{blue}{https://revistapolitica.uchile.cl}}} \vspace{1mm}

\item{\small Gonz\'alez-Bustamante, B. (2013). El estudio de las élites en Chile: Aproximaciones conceptuales y metodológicas. {\itshape Intersticios Sociales,} (6), 1--20. {\scshape url}: \href{http://ref.scielo.org/zrnp2k}{\textcolor{blue}{http://ref.scielo.org}}} \vspace{1mm}

\item{\small Gonz\'alez-Bustamante, B., \& Henr\'iquez, G. (2012). Campañas digitales: ¿Branding o participación política? {\itshape M\'as Poder Local,} (12), 32--39. {\scshape url}: \href{https://www.researchgate.net/publication/260517478_Campanas_digitales_Branding_o_participacion_politica_El_rol_de_las_redes_sociales_en_la_ultima_campana_presidencial_chilena}{\textcolor{blue}{www.researchgate.net}}} \vspace{1mm}
\end{benumerate}

\end{publications}
