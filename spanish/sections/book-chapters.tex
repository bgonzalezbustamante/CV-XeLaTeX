%%%%%%%%%%%%%%%%%%%%%%%%%%%%%%%%%%%%%%%%%%%%%%%%%%

%% CV-XeLaTeX in Spanish
%% Bastián González-Bustamante
%% University of Oxford
%% https://github.com/bgonzalezbustamante/CV-XeLaTeX

%% Based on the following repositories:
%% Awesome CV LaTeX Template for CV/Resume
%% https://github.com/posquit0/Awesome-CV
%% Bastián Gozález-Bustamante's CV-LaTeX Template
%% https://github.com/bgonzalezbustamante/CV-LaTeX
%% Carla Cisternas' CV-LuaLaTeX Template
%% https://github.com/carlacisternasg/CV-LuaLaTeX

%% Creative Commons Attribution 4.0 International License
%% https://github.com/bgonzalezbustamante/CV-XeLaTeX/blob/master/LICENSE.txt

%%%%%%%%%%%%%%%%%%%%%%%%%%%%%%%%%%%%%%%%%%%%%%%%%%

\cvsection{Capítulos en libros editados}

\begin{publications}

\begin{benumerate}{12}
\item{\small Gonz\'alez-Bustamante, B. (2018). Civil Service Models in Latin America. En A. Farazmand (ed.), {\itshape Global Encyclopedia of Public Administration, Public Policy, and Governance}. Cham: Springer. {\scshape doi}: \href{https://doi.org/10.1007/978-3-319-20928-9\_2699}{\textcolor{blue}{10.1007/978-3-319-20928-9\_2699}}. {\scshape {\footnotesize SocArXiv} doi}: \href{https://doi.org/10.31235/osf.io/mp4qd}{\textcolor{blue}{10.31235/osf.io/mp4qd}}}\vspace{1mm}

\item{\small Gonz\'alez-Bustamante, B. (2018). Internet, uso de redes sociales digitales y participación en el Cono Sur. En P. Cottet (ed.), {\itshape Opini\'on P\'ublica Contempor\'anea: Otras posibilidades de Comprensi\'on e Investigaci\'on}. Santiago: Social-Ediciones.}\vspace{1mm}

\item{\small Gonz\'alez-Bustamante, B., \& Olivares, A. (2018). La élite política gubernamental en Chile: Supervivencia de ministros. En A. Codato \& F. Espinoza (eds.), {\itshape Las \'Elites en las Am\'ericas: Diferentes Perspectivas}. Curitiba: Editora da Universidade Federal do Paraná. {\scshape url}: \href{https://www.researchgate.net/publication/325699783_Elites_en_las_Americas_diferentes_perspectivas_Elites_in_the_Americas_Different_Perspectives}{\textcolor{blue}{www.researchgate.net}}} \vspace{1mm}

\item{\small Gonz\'alez-Bustamante, B., \& Barr\'ia, D. (2018). Expansión de la esfera pública en Chile: Redes sociales, campañas electorales y participación digital. En N. Del Valle (ed.), {\itshape Transformaciones de la Esfera p\'ublica en Chile. Luchas Sociales, Espacio P\'ublico y Pluralismo Informativo}. Santiago: RIL Editores. {\scshape {\footnotesize SocArXiv} doi}: \href{https://doi.org/10.31235/osf.io/nkftb}{\textcolor{blue}{10.31235/osf.io/nkftb}}}\vspace{1mm}

\item{\small Barr\'ia, D., Gonz\'alez-Bustamante, B., \& Araya, J. E. (2017). Democracia electrónica y participación digital. Avances y desafíos. En J. R. Gil-Garc\'ia, J. I. Criado \&  J. C. T\'ellez (eds.), {\itshape Tecnolog\'ias de Informaci\'on y Comunicaci\'on en la Administraci\'on P\'ublica: Conceptos, Enfoques, Aplicaciones y Resultados}. Mexico City: INFOTEC. {\scshape url}: \href{https://www.researchgate.net/publication/321980289_Democracia_electronica_y_participacion_digital_Avances_y_desafios}{\textcolor{blue}{www.researchgate.net}}} \vspace{1mm}

\item{\small Gonz\'alez-Bustamante, B. (2016). Twitter (su uso en la comunicación política). En I. Crespo {\itshape et al.} (eds.), {\itshape Diccionario Enciclop\'edico de Comunicaci\'on Pol\'itica}. Madrid: Centro de Estudios Políticos y Constitucionales. {\footnotesize \scshape [2da edición]}.} \vspace{1mm}

\item{\small Gonz\'alez-Bustamante, B., \& Sazo, D. (2016). Campaña viral. En I. Crespo {\itshape et al.} (eds.), {\itshape Diccionario Enciclop\'edico de Comunicaci\'on Pol\'itica}. Madrid: Centro de Estudios Políticos y Constitucionales. {\footnotesize \scshape [2da edición]}.} \vspace{1mm}

\item{\small Gonz\'alez-Bustamante, B., \& Sazo, D. (2016). Imagen de marca para un candidato o gobierno (branding político). En I. Crespo {\itshape et al.} (eds.), {\itshape Diccionario Enciclop\'edico de Comunicaci\'on Pol\'itica}. Madrid: Centro de Estudios Políticos y Constitucionales. {\footnotesize \scshape [2da edición]}.} \vspace{1mm}

\item{\small Gonz\'alez-Bustamante, B. (2014). New Media and Digital Politics: The Case of Chilean Primary Presidential Elections 2013. In G. Sibaja (ed.), {\itshape Nuevos medios y comunicaci\'on pol\'itica digital}. San Jose: Fundación Educativa San Judas Tadeo. {\footnotesize \scshape [in Spanish]}.} \vspace{1mm}

\item{\small Gonz\'alez-Bustamante, B. (2014). Digital Activism, Social Networks, and Brokerage. In S. Millaleo \& P. C\'arcamo (eds.), {\itshape Mediaciones del sistema pol\'itico frente al activismo digital}. Santiago: Fundación Democracia y Desarrollo. {\footnotesize \scshape [in Spanish]}. {\scshape url}: \href{https://www.researchgate.net/publication/321992867_Activismo_digital_redes_sociales_e_intermediacion}{\textcolor{blue}{www.researchgate.net}}} \vspace{1mm}

\item{\small Gonz\'alez-Bustamante, B. (2014). Social Networks as a Tool for Political Engage: Twitter and the Public Opinion Concept. In O. Avil\'es {\itshape et al.} (eds.), {\itshape Pr\'actica Pol\'itica y Medios Digitales}. Santiago: Instituto Igualdad. {\footnotesize \scshape [in Spanish]}.} \vspace{1mm}

\item{\small Gonz\'alez-Bustamante, B., \& Henr\'iquez, G. (2013). Chile: The Digital Campaign 2009-2010. In I. Crespo \& J. del Rey (eds.), {\itshape Comunicaci\'on Pol\'itica \& Campa\~nas Electorales en Am\'erica Latina}. Buenos Aires: Editorial Biblos. {\footnotesize \scshape [in Spanish].}} \vspace{1mm}
\end{benumerate}

\end{publications}
