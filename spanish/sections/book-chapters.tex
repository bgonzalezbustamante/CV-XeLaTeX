%%%%%%%%%%%%%%%%%%%%%%%%%%%%%%%%%%%%%%%%%%%%%%%%%%

%% CV-XeLaTeX in Spanish
%% Bastián González-Bustamante
%% University of Oxford
%% https://github.com/bgonzalezbustamante/CV-XeLaTeX

%% Based on the following repositories:
%% Awesome CV LaTeX Template for CV/Resume
%% https://github.com/posquit0/Awesome-CV
%% Bastián Gozález-Bustamante's CV-LaTeX Template
%% https://github.com/bgonzalezbustamante/CV-LaTeX
%% Carla Cisternas' CV-LuaLaTeX Template
%% https://github.com/carlacisternasg/CV-LuaLaTeX

%% LaTeX Project Public License v1.3c
%% https://github.com/bgonzalezbustamante/CV-XeLaTeX/blob/master/LICENSE.md

%%%%%%%%%%%%%%%%%%%%%%%%%%%%%%%%%%%%%%%%%%%%%%%%%%

\cvsection{Capítulos de libros}

\begin{publications}

\begin{benumerate}{12}

\item{\small González-Bustamante, B. (2018). Civil Service Models in Latin America. En A. Farazmand (ed.), {\itshape Global Encyclopedia of Public Administration, Public Policy, and Governance}. Cham: Springer. {\scshape doi}: \href{https://doi.org/10.1007/978-3-319-20928-9\_2699}{\textcolor{blue}{10.1007/978-3-319-20928-9\_2699}}. {\scshape \footnotesize SocArXiv:} \href{https://doi.org/10.31235/osf.io/mp4qd}{\textcolor{blue}{10.31235/osf.io/mp4qd}}.}\vspace{1mm}

\item{\small González-Bustamante, B., \& Olivares, A. (2018). La élite política gubernamental en Chile: Supervivencia de ministros. En A. Codato \& F. Espinoza (eds.), {\itshape Las Élites en las Américas: Diferentes Perspectivas}. Curitiba: Editora da Universidade Federal do Paraná. {\scshape url:} \href{https://www.researchgate.net/publication/325699783_Elites_en_las_Americas_diferentes_perspectivas_Elites_in_the_Americas_Different_Perspectives}{\textcolor{blue}{www.researchgate.net}}.}\vspace{1mm}

\item{\small González-Bustamante, B., \& Barría, D. (2018). Expansión de la esfera pública en Chile: Redes sociales, campañas electorales y participación digital. En N. Del Valle (ed.), {\itshape Transformaciones de la Esfera Pública en Chile. Luchas Sociales, Espacio Público y Pluralismo Informativo}. Santiago: RIL Editores. {\scshape \footnotesize SocArXiv:} \href{https://doi.org/10.31235/osf.io/nkftb}{\textcolor{blue}{10.31235/osf.io/nkftb}}.}\vspace{1mm}

\item{\small González-Bustamante, B. (2018). Internet, uso de redes sociales digitales y participación en el Cono Sur. En P. Cottet (ed.), {\itshape Opinión Pública Contemporánea: Otras posibilidades de Comprensión e Investigación}. Santiago: Social-Ediciones. {\scshape doi:} \href{https://doi.org/10.34720/2nd0-8t73}{\textcolor{blue}{10.34720/2nd0-8t73}}.}\vspace{1mm}

\item{\small Barría, D., González-Bustamante, B., \& Araya, J. E. (2017). Democracia electrónica y participación digital. Avances y desafíos. En J. R. Gil-García, J. I. Criado \&  J. C. Téllez (eds.), {\itshape Tecnologías de Información y Comunicación en la Administración P\'ublica: Conceptos, Enfoques, Aplicaciones y Resultados}. Ciudad de México: INFOTEC. {\scshape url:} \href{https://www.researchgate.net/publication/321980289_Democracia_electronica_y_participacion_digital_Avances_y_desafios}{\textcolor{blue}{www.researchgate.net}}.}\vspace{1mm}

\item{\small González-Bustamante, B., \& Sazo, D. (2016). Campaña viral. En I. Crespo {\itshape et al.} (eds.), {\itshape Diccionario Enciclopédico de Comunicación Política}. Madrid: Centro de Estudios Políticos y Constitucionales. {\footnotesize \scshape [2da edición]}.}\vspace{1mm}

\item{\small González-Bustamante, B., \& Sazo, D. (2016). Imagen de marca para un candidato o gobierno (branding político). En I. Crespo {\itshape et al.} (eds.), {\itshape Diccionario Enciclopédico de Comunicación Política}. Madrid: Centro de Estudios Políticos y Constitucionales. {\footnotesize \scshape [2da edición]}.}\vspace{1mm}

\item{\small González-Bustamante, B. (2016). Twitter (su uso en la comunicación política). En I. Crespo {\itshape et al.} (eds.), {\itshape Diccionario Enciclopédico de Comunicación Política}. Madrid: Centro de Estudios Políticos y Constitucionales. {\footnotesize \scshape [2da edición]}.}\vspace{1mm}

\item{\small González-Bustamante, B. (2014). Nuevos medios y política digital: El caso de las primarias presidenciales chilenas de 2013. En G. Sibaja (ed.), {\itshape Nuevos Medios y Comunicaci\'on Pol\'itica Digital}. San José: Fundación Educativa San Judas Tadeo.}\vspace{1mm}

\item{\small González-Bustamante, B. (2014). Activismo digital, redes sociales e intermediación. En S. Millaleo \& P. C\'arcamo (eds.), {\itshape Mediaciones del Sistema Pol\'itico frente al Activismo Digital}. Santiago: Fundación Democracia y Desarrollo. {\scshape url:} \href{https://www.researchgate.net/publication/321992867_Activismo_digital_redes_sociales_e_intermediacion}{\textcolor{blue}{www.researchgate.net}}}\vspace{1mm}

\item{\small González-Bustamante, B. (2014). Redes sociales como herramienta para la práctica política: Twitter y el concepto de opinión pública. En O. Avilés {\itshape et al.} (eds.), {\itshape Práctica Política y Medios Digitales}. Santiago: Instituto Igualdad.}\vspace{1mm}

\item{\small González-Bustamante, B., \& Henríquez, G. (2013). Chile: La campaña digital 2009-2010. En I. Crespo \& J. del Rey (eds.), {\itshape Comunicación Política \& Campa\~nas Electorales en América Latina}. Buenos Aires: Editorial Biblos.}\vspace{1mm}

\end{benumerate}

\end{publications}
